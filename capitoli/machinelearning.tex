% !TEX encoding = IsoLatin 

% Affinch� gli accenti vengano accettati, assicurati che la codifica di questo file
% sia ISO 8859-1

% PER OTTENERE IL PDF, digitare da terminale
% ./makepdfplease
% 


\section{Machine Learning}
L'approccio del Machine Learning consiste nell'usare un algoritmo per produrre un modello che minimizzi 
gli errori di classificazione rispetto a delle classificazioni vere per ipotesi.
Questo ha una conseguenza importante: un classificatore, una volta istruito a riconoscere una classe
partendo da una certa quantit� di esempi giusti, potr� stimare la classe anche in presenza di dati 
non uguali agli esempi di partenza.
Ci� costituisce una grande differenza rispetto alla regex, che consente di rilevare un testo
solo se esattamente conforme all'automa da essa descritto.

\section{Annotazioni gold in Doccano: manuale dell'annotatore}
Per chi si occuper� dell'annotazione � necessario sapere che i dati SOA non sono scritti in maniera uguale in tutti i documenti.
Vogliamo dare un'idea di quanti modi diversi esistono per indicare una categoria: per esempio OG-12, \say{Opere ed impianti di bonifica e protezione ambientale}, pu� essere indicato in uno qualsiasi dei modi seguenti:
\begin{itemize}
\item Opere Generali 12;
\item Opere Generali di tipo 12;
\item Op.Gen. cat.12;
\item og 12;
\item og12;
\item o.g. 12;
\item og12;
\item o.g.12;
\item og.12;
\item og. 12;
\item og-12;
\item o.g.-12;
\item og.-12;
\item OG 12, "Opere ed impianti di bonifica e protezione ambientale".
\end{itemize}
Se si aggiunge che alcuni dei documenti sono stati digitalizzati a partire da fogli stampati e acquisiti con OCR, ci potrebbero essere degli errori di interpretazione, con conseguenze del genere:
0G12 (carattere di \say{zero} al posto della lettera \say{O});
0Gl2 (\say{L}  minuscola al posto del carattere numerico \say{1}).
Casi di errore cos� banali non lasciano dubbi sul significato (\say{OG-12}) per cui qualsiasi umano pu� correttamente annotarli.

� altrettanto possibile che la categoria economica, normalmente espressa in numeri romani, si trovi scritta in modi differenti; vediamo il caso della III categoria, di seguito:\begin{itemize}
\item categoria III, con importo xyz;
\item cat.III;
\item categoria terza; 
\item cat.III�;
\item cat. 3�.
\end{itemize}
Anche le categorie possono soffrire di errori OCR: � frequente ad esempio che la \say{l} prenda il posto della \say{I}, ma chi annota non avr� problemi a riconoscere il significato dell'informazione.


\section{La libreria SpaCy}
Per l'implementazione di una rete neurale abbiamo fatto uso della libreria \textbf{Spacy}, una libreria \text it{open-source} scritta in Python e Cython che implementa funzionalit� di Natural Language Processing. Tra i progetti che appartengono al mondo SpaCy vi � la libreria Thinc, che permette di importare modelli statistici da PyTorch, TensorFlow e MXNet\cite{spacyThinc}. Spacy fornisce funzionalit� di \textbf{tagger}, \textbf{parser}, \textbf{text categorizer}, \textbf{ner} e permette di articolarli in una pipeline; inoltre rende possibile la configurazione di nuovi componenti e la loro aggiunta alla pipeline. 

\section{Uso di Spacy}
Per l'uso di SpaCy abbiamo dovuto creare gli insiemi di Training, Development e Test.
Le annotazioni gold della ground truth sono state dunque divise in questi tre insiemi, 
rispettivamente aventi il 70\%, il 10 \% e il 20 \% delle annotazioni golden.

\paragraph{Configurazione della pipeline:}
Istruisco SpaCy su quali sono le funzionalit� da inserire in pipeline. Nel caso in questione la pipeline deve effettuare la tokenizzazione di elementi testuali della lingua italiana e la loro classificazione come entit�. A livello di configurazione questo si traduce in queste righe del file base\_config.cfg:
\begin{lstlisting}[language=Python]
[nlp]
lang = "it"
pipeline = ["tok2vec","ner"]
\end{lstlisting}
Basta poi il seguente comando per ottenere la configurazione completa della pipeline:
\begin{lstlisting}[language=Python]
spacy init fill-config base_config.cfg config.cfg
\end{lstlisting}

\paragraph{Fase di apprendimento:}
La costruzione del modello adopera il le golden labels di training e viene svolta col seguente comando:
\begin{lstlisting}[language=Python]
spacy train config.cfg --paths.train ./train.spacy --paths.dev ./dev.spacy
  --output ./output
\end{lstlisting}

\paragraph{Debug:}
Data un'entit� \textit{E}, avere un numero di esempi di \textit{E} troppo limitato significherebbe dare poca esperienza di \textit{E} al modello.Per questo SpaCy fornisce un comando che controlla il numero di esempi per ogni entit�, effettuando al contempo dei controlli ortografici su tutte le labels:
\begin{lstlisting}[language=Python]
python3 -m spacy debug data config.cfg --paths.train ./train.spacy 
  --paths.dev ./dev.spacy
\end{lstlisting}

\paragraph{Valutazione di precision e recall:}
Una volta istruito un modello ed esserci assicurati che le annotazioni che usa sono valide,
possiamo mettere alla prova il modello: sottoponiamo al modello il docbin di test, le cui entit� sono gi� annotate,
e confrontiamo l'output del modello con le annotazioni gold.
Tale confronto viene eseguito da Spacy con il seguente comando:
\begin{lstlisting}[language=Python]
spacy evaluate output/model-best/ test.spacy --displacy-path . --displacy-limit 100  
  --output output.json 
\end{lstlisting}


