% !TEX encoding = IsoLatin 

% Affinch� gli accenti vengano accettati, assicurati che la codifica di questo file
% sia ISO 8859-1

% PER OTTENERE IL PDF, digitare da terminale
% ./makepdfplease
% 

% !TEX encoding = IsoLatin 

% Affinch� gli accenti vengano accettati, assicurati che la codifica di questo file
% sia ISO 8859-1

% PER OTTENERE IL PDF, digitare da terminale
% ./makepdfplease
% 
% !TEX encoding = IsoLatin 

% Affinch� gli accenti vengano accettati, assicurati che la codifica di questo file
% sia ISO 8859-1

% PER OTTENERE IL PDF, digitare da terminale
% ./makepdfplease
% 



\chapter{Estrazione di dati da testi della pubblica amministrazione}
%\chapter[Estrazione di dati da testi della pubblica amministrazione][]{1. Estrazione di dati da testi della pubblica amministrazione}
%IoT \cite{friis} basically refers to ....\\
%\ac{IoT} basically refers to ....\\

%\section{Introduzione}


%\section{Contesto}


Tra le funzioni di uno Stato vi sono la costruzione, la manutenzione e la messa in sicurezza delle infrastrutture di pubblica utilit�, quali ad esempio gli edifici pubblici, le reti di telecomunicazione, le strade e gli ospedali di cui beneficiano i cittadini.
La \acrlong{Pubblica Amministrazione}, dovendo garantire una quantit� cos� grande di infrastrutture, non esegue i lavori direttamente, ma ne delega l'attuazione pratica a delle imprese private scelte con una pubblica gara d'appalto.
Nel presente capitolo partiamo dallo scenario degli appalti per introdurre i dati SOA, che saranno oggetto della 
nostra ricerca.

\section{Gare d'appalto}
L'appalto � un contratto con il quale una parte, detta parte appaltatrice, assume, \say{con organizzazione dei mezzi necessari e con gestione a proprio rischio, l'obbligazione di compiere in favore di un'altra (committente o appaltante) un'opera o un servizio}\cite{basacchi2013codice}.
Gli appalti pubblici sono dunque un modo per eseguire delle opere pubbliche pagando delle imprese private per la realizzazione vera e propria; la \acrshort{Pubblica Amministrazione} deve per� assicurarsi che:
l'impresa abbia le conoscenze per portare a termine l'opera rispettando degli standard tecnici, ossia seguendo la \textbf{regola d'arte};
l'impresa gestisca i lavori in maniera competitiva, senza comportare sprechi ingiustificati di soldi pubblici;
l'impresa riesca a reggere lo sforzo finanziario che l'opera comporta, sia per quanto riguarda l'approvvigionamento dei materiali, sia per quanto riguarda il costo del lavoro.
L'assegnazione di un progetto si avvale di apposita gara d'appalto, che ha come obbiettivo la selezione dell'impresa che meglio possa soddisfare questi requisiti di competenze tecniche, finanziarie e di competitivit� sul mercato.

\section{Identificativi di progetto e di gara}
Nell'ambito di una gara d'appalto, il progetto pu� essere composto da pi� parti singole chiamate lotti; l'assegnazione di un lotto � indipendente dagli altri, per cui diversi lotti di un progetto possono essere assegnati ad aziende differenti nell'ambito della stessa gara d'appalto.
I documenti che andiamo ad analizzare riportano una serie di informazioni di nostro interesse, quali:
\begin{itemize}
\item il \textbf{\acrfull{Codice Unico di Progetto}}, che descrive univocamente il progetto d'investimento pubblico;  � un identificativo alfanumerico di quindici caratteri;
\item il \textbf{\acrfull{Codice Identificativo di Gara}}, che indica in maniera univoca il lotto;
 � un identificativo alfanumerico di dieci caratteri;
\end{itemize}

\section{Lavori e importi}
La gara d'appalto deve indicare in maniera evidente e inoppugnabile quali competenze tecniche e quali capacit� finanziarie sono ritenute requisiti fondamentali per la partecipazione; qualsiasi ambiguit� nei requisiti di una gara potrebbe dare adito a ricorsi e quindi a rallentare la gara stessa. 
Il bisogno di requisiti oggettivi ha motivato l'introduzione di apposite certificazioni fornite dalle 
\acrfull{societa-soa}, che prendono il nome di \say{certificati SOA} o \say{attestazioni SOA}.
Tali attestazioni si dividono in due macrogruppi:
\begin{itemize}
\item \textbf{Categorie di opere:} individuano le categorie di lavori dal punto di vista tecnico;
\item \textbf{Classifiche di importi:} individuano i livelli di capacit� finanziaria.
\end{itemize}
Un bando di gara esprimer� i requisiti di progetto sotto forma di categorie e classifiche SOA, per cui un'azienda che voglia  essere ammessa alla gara dovr� possedere le attestazioni SOA richieste.









\section{Tassonomia SOA}
Le Categorie sono suddivise in due macro-categorie: opere generali e opere specializzate,  rispettivamente identificate dagli acronimi \say{OG} e \say{OS}; sono individuate 13 categorie di Opere Generali e 39 categorie di Opere Specializzate.
Le Classifiche d'importo, in numero di dieci, sono rese come numeri ordinali romani. 


\subsection{Categorie di opere generali}\label{CATEGORIE}
\begin{itemize}
  \item \acrshort{OG-1}, \acrlong{OG-1}
  \item \acrshort{OG-2}, \acrlong{OG-2}
\item \acrshort{OG-3}, \acrlong{OG-3}
\item \acrshort{OG-4}, \acrlong{OG-4}
\item \acrshort{OG-5}, \acrlong{OG-5}
\item \acrshort{OG-6}, \acrlong{OG-6}
\item \acrshort{OG-7}, \acrlong{OG-7}
\item \acrshort{OG-8}, \acrlong{OG-8}
\item \acrshort{OG-9}, \acrlong{OG-9}
\item \acrshort{OG-10}, \acrlong{OG-10}
  \item \acrshort{OG-11}, \acrlong{OG-11}
  \item \acrshort{OG-12}, \acrlong{OG-12}
  \item \acrshort{OG-13}, \acrlong{OG-13}
\end{itemize}

\subsection{Categorie di opere specializzate}
\begin{itemize}
  \item \acrshort{OS-1}, \acrlong{OS-1}
  \item \acrshort{OS-2-A}, \acrlong{OS-2-A}
  \item \acrshort{OS-2-B}, \acrlong{OS-2-B}
  \item \acrshort{OS-3}, \acrlong{OS-3}
  \item \acrshort{OS-4}, \acrlong{OS-4}
  \item \acrshort{OS-5}, \acrlong{OS-5}
  \item \acrshort{OS-6}, \acrlong{OS-6}
  \item \acrshort{OS-7}, \acrlong{OS-7}
  \item \acrshort{OS-8}, \acrlong{OS-8}
  \item \acrshort{OS-9}, \acrlong{OS-9}
  \item \acrshort{OS-10}, \acrlong{OS-10}
  \item \acrshort{OS-11}, \acrlong{OS-11}
  \item \acrshort{OS-12-A}, \acrlong{OS-12-A}
  \item \acrshort{OS-12-B}, \acrlong{OS-12-B}
  \item \acrshort{OS-13}, \acrlong{OS-13}
  \item \acrshort{OS-14}, \acrlong{OS-14}
  \item \acrshort{OS-15}, \acrlong{OS-15}
  \item \acrshort{OS-16}, \acrlong{OS-16}
  \item \acrshort{OS-17}, \acrlong{OS-17}
  \item \acrshort{OS-18-A}, \acrlong{OS-18-A}
  \item \acrshort{OS-18-B}, \acrlong{OS-18-B}
  \item \acrshort{OS-19}, \acrlong{OS-19}
  \item \acrshort{OS-20-A}, \acrlong{OS-20-A}
  \item \acrshort{OS-20-B}, \acrlong{OS-20-B}
  \item \acrshort{OS-21}, \acrlong{OS-21}
  \item \acrshort{OS-22}, \acrlong{OS-22}
  \item \acrshort{OS-23}, \acrlong{OS-23}
  \item \acrshort{OS-24}, \acrlong{OS-24}
  \item \acrshort{OS-25}, \acrlong{OS-25}
  \item \acrshort{OS-26}, \acrlong{OS-26}
  \item \acrshort{OS-27}, \acrlong{OS-27}
  \item \acrshort{OS-28}, \acrlong{OS-28}
  \item \acrshort{OS-29}, \acrlong{OS-29}
  \item \acrshort{OS-30}, \acrlong{OS-30}
  \item \acrshort{OS-31}, \acrlong{OS-31}
  \item \acrshort{OS-32}, \acrlong{OS-32}
  \item \acrshort{OS-33}, \acrlong{OS-33}
  \item \acrshort{OS-34}, \acrlong{OS-34}
  \item \acrshort{OS-35}, \acrlong{OS-35}
\end{itemize}

\subsection{Classifiche di importi }\label{CLASSIFICHE}
\begin{itemize}
  \item \acrshort{I}, \acrlong{I}
  \item \acrshort{II}, \acrlong{II}
  \item \acrshort{III}, \acrlong{III}
  \item \acrshort{III-bis}, \acrlong{III-bis}
  \item \acrshort{IV}, \acrlong{IV}
  \item \acrshort{IV-bis}, \acrlong{IV-bis}
  \item \acrshort{V}, \acrlong{V}
  \item \acrshort{VI}, \acrlong{VI}
  \item \acrshort{VII}, \acrlong{VII}
  \item \acrshort{VIII}, \acrlong{VIII}
\end{itemize}  
