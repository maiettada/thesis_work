% !TEX encoding = IsoLatin 

% Affinch� gli accenti vengano accettati, assicurati che la codifica di questo file
% sia ISO 8859-1

% PER OTTENERE IL PDF, digitare da terminale
% ./makepdfplease
% 




\section{Impiego delle regular expression}
\subsection{Difficolt� insite nelle regex}
Le espressioni regolari sono usatissime nel mondo della programmazione: 
si stima che siano utilizzate in \say{pi� di un terzo dei progetti Python e Javascript}[cita R.A.H.].
La loro grande diffusione non vuol dire per� che costituiscano una tecnologia sicura da utilizzare, poich� presentano
problematiche di leggibilit�, scarsit� di documentazione, difficile manutenibilit� e problemi di sicurezza non da poco conto.
\paragraph{Leggibilit�: } un grave problema delle regex � che, rappresentando dei pattern alfanumerici sintetizzati all'estremo, risultino incomprensibili alla lettura; gli sviluppatori non sono immuni a questo problema ed anche per 
loro risalire al pattern alfanumerico partendo da una regex pu� essere un compito scomodo.
\paragraph{Documentazione: } nel mondo della programmazione � consolidata la pratica di documentare i passaggi meno 
leggibili di un algoritmo; nel caso delle regular expression, queste sono inserite negli algoritmi sotto forma di stringhe e ci� pu� influenzare il modo in cui gli sviluppatori le documentano:
\begin{itemize}
\item in alcuni IDE � disponibile la syntax highlighting per evidenziare i diversi "componenti" della regex;
\item alcuni sviluppatori trovano comodo "spezzare" la regex su pi� linee di codice;
questo consentirebbe di documentare linea per linea ogni "componente" della regex, ma non tutti i linguaggi lo permettono.
\end{itemize}
Infine molti sviluppatori esperti considerano le regex come "self-documenting"[cita R.A.H.] e si rifiutano dunque di documentarle, contribuendo a renderle problematiche dal punto di vista della documentazione.
\paragraph{Manutenibilit�: } quando la leggibilit� � scarsa e la documentazione � insufficiente, � giocof�rza che 
qualsiasi modifica apportata a una regex possa, oltre ad includere nuovi pattern desiderati, escludere erroneamente
e involontariamente tutta una serie di pattern che prima erano correttamente rilevati.
\paragraph{Denial Of Service: } l'algoritmo di Spencer con cui sono implementate molte \textbf{regex engine}, soffre di una
worst-case time complexity che pu� diventare esponenziale in alcuni casi di NFA non deterministici[cita regex engines]. Sebbene tale vulnerabilit� esponga i server ad attacchi di \textbf{ReDoS}, gli sviluppatori software spesso non hanno la minima formazione su tale problema.


\subsection{Regex: quando conviene usarle}
Abbiamo visto finora quanto le regex possano essere un buono strumento software, ma anche presentare problematiche molto
critiche; la comunit� dei programmatori ha persino coniato il detto \textit{"Now you have two problems"}[cita], che esprime come le soluzioni software basate su regex, lungi dall'essere considerate affidabili, creino ulteriori problemi 
a causa della gestione delle regex stesse.
Nello studio \textit{"How to invest my time"} [cita HtimT] gli autori, ben consapevoli di quanto le regex siano complesse 
ed error-prone, si domandano \textit{fino a che punto} possano essere usate vantaggiosamente;
pi� in particolare il loro studio si cala nel contesto della Entity Extraction e si pone l'obiettivo di usare al m
meglio le risorse umane, studiando due attivit� diverse e complementari:
\begin{itemize}
\item lo sviluppo di regex;
\item l'annotazione manuale di dati.
\end{itemize}
Entrambe le attivit� umane sono volte all'addestramento di una rete neurale: la prima, producendo annotazioni con l'uso 
della regex; la seconda, lasciando all'operatore umano la creazione di annotazioni, che possono essere usate sia per l'addestramento della rete neurale, sia per un successivo fine-tuning della rete neurale.
I risultati sperimentali di questo studio mostrano che:
\begin{itemize}
\item se il tempo da investire nella EE � poco (inferiore ai 40 minuti), conviene che l'operatore si limiti a produrre una regex;
\item se il tempo � molto (superiore ai 40 minuti), l'operatore potrebbe spendere tutto il tempo a sua disposizione per creare annotazioni con cui istruire la rete neurale;
\item tra i due casi estremi, pu� convenire che l'operatore umano spenda pochi minuti per creare una regex per un primo setup di rete neurale, per poi aggiungervi annotazioni manuali per farne fine-tuning.
\end{itemize}
Questo approccio che contempla le azioni umane in un sistema automatico da addestrare e perfezionare � detto \textbf{Human In The Loop}.  

%\subsection{ML-based}


%\subsection{human in the loop}
