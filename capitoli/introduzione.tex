% !TEX encoding = IsoLatin 

% Affinch� gli accenti vengano accettati, assicurati che la codifica di questo file
% sia ISO 8859-1

% PER OTTENERE IL PDF, digitare da terminale
% ./makepdfplease
% 

\chapter{Introduzione}
L'avanzamento tecnologico delle telecomunicazioni negli ultimi decenni ha permesso alle persone di recuperare dati e informazioni provenienti dai luoghi pi� disparati e ad una velocit� di approvvigionamento praticamente nulla: se prima di Internet l'unico modo di ricevere un documento ufficiale era farne richiesta all'ufficio competente, ora le documentazioni possono essere normalmente a disposizione in rete. Gli utenti della rete hanno a disposizione una mole cos� grande di documenti e informazioni che gli � praticamente impossibile visionare la totalit� di questi dati.
In questo contesto l'esigenza di estrarre dati da documenti � sempre pi� sentita, soprattutto quando si voglia fare una cernita dei documenti contenenti specifiche informazioni di proprio interesse.
Questa tesi ha come scopo effettuare un'estrazione di dati da un insieme di documenti della \acrfull{Pubblica Amministrazione}; pi� precisamente i documenti appartengono al contesto dei bandi pubblici e i dati da cercare sono tutte quelle informazioni che impongono dei requisiti ai possibili partecipanti di gara. 
Il lavoro � strutturato come segue: al capitolo secondo si descrivono i bandi pubblici e dati SOA in essi contenuti; 
al capitolo terzo si presenta il problema della ricerca di tali dati formalizzata come un problema di Named Entity Extraction.
Al capitolo quarto viene svolta una rassegna di lavori correlati, che vertono sulle metodologie di estrazione di dati; 
dopodich� tali metodologie vengono implementate al capitolo quinto. 
In questo contesto si provano approcci tecnologici diversi, nella fattispecie l'uso delle regular expressions e delle reti neurali. Di entrambe le tecnologie vengono delineati punti di forza e di debolezza, per poi investigare se sia preferibile l'una delle due o se invece sia possibile un compromesso che valorizzi il meglio di entrambe.
Al capitolo sesto si introduce un approccio ibrido configurato come Human In The Loop, che viene proposto come 
soluzione vantaggiosa rispetto alle tecnologie singole considerate inizialmente. Nell'ultimo capitolo sono esposte le conclusioni e le possibilit� di sviluppi successivi per questo lavoro di ricerca.