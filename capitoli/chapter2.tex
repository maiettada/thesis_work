% !TEX encoding = IsoLatin 

% Affinch� gli accenti vengano accettati, assicurati che la codifica di questo file
% sia ISO 8859-1

% PER OTTENERE IL PDF, digitare da terminale
% ./makepdfplease
% 





\section{Casi d'uso reali}
\subsection{}
Finora sono state descritte le attestazioni SOA ed � stato chiarito il ruolo fondamentale che queste certificazioni ricoprono negli appalti.
D'ora in avanti si intende descrivere l'uso di queste attestazioni da parte dei vari attori del mondo degli appalti:
\begin{enumerate}
  \item la \textbf{Pubblica Amministrazione}: produce il bando di gara, relativo ad un progetto identificato univocamente dal codice CUP ed eventualmente diviso in lotti, ognuno identificato da un codice CIG; nel bando di gara aggiunge i requisiti di Categorie SOA e le Classifiche economiche SOA, imponendo eventualmente vincoli temporali su tali certificati (?L'azienda deve risultare in possesso di tale certificato dal 2020?); il bando viene scritto in formato pdf e pubblicato sul sito dell'ente pubblico ( Comune, Provincia, Regione, Ministero, et cetera), che risulta in questo contesto essere l'appaltante;
\item l'\textbf{imprenditore}, o qualsivoglia candidato appaltatore: naviga i siti web della Pubblica Amministrazione alla ricerca di bandi di gara; per ogni bando di gara trovato, deve comprendere i requisiti SOA e accertarsi di poterli soddisfare; solo se dotato di idonea attestazione, potr� candidarsi ad essere appaltatore prendendo parte alla gara d'appalto.
\end{enumerate}
Sia chiaro che la P.A. pubblica anche altre tipologie di documenti, per cui � utile precisare che ci riferiamo a quell'insieme di documenti che descrivono gare e verbali d'appalto:  possiamo riferirci a questo insieme chiamandolo dominio d'appalto.
In questi due casi d'uso possiamo evidenziare una prima problematica: ogni ente pubblico ha un proprio sito web e la raccolta di documenti di appalti pu� essere un'attivit� lunga e dispersiva. Questo problema potrebbe essere risolto da un software di tipo web-crawler che navighi i siti web della pubblica amministrazione e raccolga tutti i documenti del dominio d'appalto in un dataset.
Esiste una seconda problematica da porre: ammesso che abbia a disposizione tutto il dataset del dominio degli appalti, l'imprenditore che � alla ricerca di un bando adatto alla sua specifica certificazione si trover� costretto a verificare per ogni documento se tale certificazione sia sufficiente per essere ammesso al bando; detto in altre parole, dovr� leggere e annotare manualmente tutti i documenti in base alle attestazioni SOA per poi scegliere il bando con l'annotazione SOA desiderata.
Questa problematica � decisamente noiosa, sia perch� la lettura toglie alla persona un ingente tempo di lavoro, sia perch� un compito del genere � perfettamente automatizzabile; d'ora in poi ci focalizzeremo su questa attivit� specifica.
L'idea di base � permettere alle imprese di migliorare la ricerca di possibili appalti rendendo questi documenti digitalmente navigabili in base alle attestazioni SOA: 
per l'imprenditore il caso d'uso 2 si ridurrebbe a cercare l'attestazione SOA desiderata per avere tutti e soli i bandi che la contemplino tra i requisiti.
La ricerca di attestazioni SOA in un testo � un problema di \textbf{Entity Extraction} (\textbf{EE}).
\paragraph{Formulazione del problema:}
\textit{
Dato un insieme di documenti in formato testuale appartenenti al dominio delle gare d'appalto, estrarre le attestazioni SOA ivi menzionate.
}

\section{Impostazione del problema}
\subsection{Insiemi di valori}
L'impostazione del problema inizia con la definizione degli insiemi di valori possibili;
nel caso delle attestazioni SOA, seguendo la tassonomia esposta definiamo l'insieme delle Categorie e l'insieme delle Classifiche:
\begin{center}
Categorie = \{\say{OG-1}, \say{OG-2}, \say{OG-3}, \say{OG-4},
 \say{OG-5},\say{OG-6},\say{OG-7},\say{OG-8},
 \say{OG-9},\say{OG-10},\say{OG-11},\say{OG-12},
 \say{OG-13}, \say{OS-1},\say{OS-2A},\say{OS-2B},
 \say{OS-3},
 
 \say{OS-4},\say{OS-5},\say{OS-6},
 \say{OS-7},\say{OS-8},\say{OS-9},\say{OS-10},
 \say{OS-11},\say{OS-12A},
 
 \say{OS-12B}, \say{OS-13},
 \say{OS-14},\say{OS-15},\say{OS-16},\say{OS-17},
 \say{OS-18A},\say{OS-18B},
 
 \say{OS-19},
 \say{OS-20A},
 \say{OS-20B},\say{OS-21},\say{OS-22},\say{OS-24},
 \say{OS-25}, \say{OS-26}, 
 
 \say{OS-27}, \say{OS-28},
 \say{OS-29}, \say{OS-30}, \say{OS-31}, \say{OS-32},
 \say{OS-33}, \say{OS-34}, 
 \say{OS-35} \}
\end{center}

\begin{center}
Classifiche = \{\say{I}, \say{II}, \say{III}, \say{IV}, \say{IV-bis}, \say{V}, \say{VI}, \say{VII}, \say{VIII} \};
\end{center}

In definitiva, il sistema indicher� le informazioni estratte in base all'insieme Categorie e Classifiche cos� definiti. 
\subsection{Formato dei dati estratti}
Pi� in generale, per ogni attestazione SOA che il sistema individuer� in un documento, dovr� indicare varie informazioni:
\begin{itemize}
    \item la porzione di testo in cui l'attestazione SOA � individuata;
    \item la categoria riconosciuta;
    \item la classifica riconosciuta;
    \item lo score, ossia il grado di accuratezza dell'output.
\end{itemize}

\paragraph{soa\_extracted\_tuple} = [  (start\_offset\_cat, end\_offset\_cat), (categoria, classifica), score)
            ]

\section{Valutazione dei dati estratti}
Quando il sistema collega la stringa "o.g. 1" all'elemento OG-1, sta effettuando una classificazione: sta classificando un dato come appartenente alla classe delle istanze di OG-1.
Il sistema di estrazione dati effettua una classificazione su ogni porzione di testo individuata; ogni porzione di testo pu� essere collegata alla classe di uno dei valori SOA.
Tipicamente, in un problema di classificazione l'output pu� essere valutato come segue:
\begin{itemize}
    \item True Positive(TP), se l'elemento � stato assegnato correttamente alla classe;
    \item False Positive(FP), se l'elemento � stato assegnato erroneamente alla classe;
    \item True Negative(TN), se l'elemento non � stato assegnato alla classe perch� non vi andava assegnato;
    \item False Negative(FN), se l'elemento non � stato assegnato alla classe ma vi andava assegnato.
\end{itemize}

Ogni output del sistema verr� valutato in termini di TP-FP-TN-FN; questo permetter� di calcolare la bont� del sistema in termini di:

\paragraph{Precision} = (TP)/(TP + FP )
\paragraph{Recall} = (TP)/(TP + FN )
\paragraph{Accuracy} = (TP + TN)/(TP + TN + FP + FN)

\section{Ground truth}
Per ogni istanza di testo assegnato ad una classe, bisogna affermare se la classificazione data costituisce true positive, false positive, true negative o false negative; questo vuol dire che, per valutare la bont� di una classificazione, abbiamo bisogno di stabilire quale sia la classificazione giusta.
Il sistema che stiamo considerando effettua un'estrazione di dati, classificandoli come elementi delle attestazioni SOA; per valutare la bont� del sistema abbiamo dunque bisogno di stabilire per ogni porzione di testo l'output giusto, ovvero di definire la \textbf{verit�} in base alla quale valutare il sistema.
La \textbf{Ground Truth} � dunque definita come l'output corretto che ci si aspetterebbe dal sistema; e come tale indicher�, per ogni porzione di documento, l'attestazione "giusta" che il sistema avrebbe dovuto estrarre.
La ground truth verr� poi confrontata con l'output del sistema e permetter� di classificare ogni tupla dell'output come true positive, false positive, true negative, false negative.
Una volta contrassegnata ogni tupla dell'output come tp/fp/tn/fn, sar� possibile calcolare Precision, Recall e Accuracy.

\section{Tecniche di estrazione}
Di seguito si espongono due approcci principali per l'estrazione di dati:
\begin{itemize}
    \item l'uso di regular expression;
    \item l'uso di tecniche di machine learning.
\end{itemize}
Mostreremo che hanno differenti punti di forza, differenti punti di debolezza e confronteremo i risultati di entrambe le tecnologie.