% !TEX encoding = IsoLatin 

% Affinch� gli accenti vengano accettati, assicurati che la codifica di questo file
% sia ISO 8859-1

% PER OTTENERE IL PDF, digitare da terminale
% ./makepdfplease
% 
\chapter[Valutazione dei risultati][]{5. Valutazione dei risultati}

Dopo l'implementazione dei prototipi NER-regex e NER-ML, possiamo introdurre la loro valutazione con l'uso di un apposito Scorer.
Dato un sistema NER, lo Scorer fornisce delle misure di affidabilit� dell'NER stesso in termini di \textbf{precision}, \textbf{recall} e \textbf{accuracy}.



\section{Confronto tra regex e ML}
Nella tabella scores-comparisons.pdf\ref{fig:v1-nums} riportiamo i valori di Precision, Recall e Accuracy ottenuti per ogni entit� contemplata nel task di classificazione, sia nel caso di NER-regex che nel caso NER-ML.
Le regex considerate sono le seguenti:
 \[regex-soa\sb{wl-v1}=\textsc{\char13}O(S|G)(\\d\\d?)\textsc{\char13} \]  \[ regex-class\sb{wl-v1}=\textsc{\char13}(IV|IV|III|II|IX|VIII|VII|VI|X|V|I)\textsc{\char13} \]



Si pu� facilmente notare come molte entit� di tipo \textbf{Categoria} vedono valori di precision recall e accuracy migliori nell'approccio Machine Learning; in particolare sono stati evidenziati in verde i valori di precision, recall ed f1-score che in NER-ML superano almeno dello 0.05 i corrispondenti valori registrati dall'NER-regex. 
In un caso (entity type OG-4) si registra un peggioramento delle performance del Machine Learning, che abbatte a zero le score.
Questo pu� essere spiegato con la mancanza di esempi numerosi per alcune entit�: 
ci� porterebbe il modello a funzionare molto bene nel riconoscere le entit� ben rappresentate dalla ground truth,
ma anche a funzionare peggio per tutte le entit� meno note.

\includegraphics[trim=0 300 0 0, clip, width=\textwidth,height=\textheight,keepaspectratio, page=1]{ConfrontiNumerici+ - v1.pdf} 
\captionof{figure}{confronto tra l'approccio regex(sinistra) e NN(destra)}
\label{fig:v1-nums}

Come sottolineato precedentemente, l'approccio NER-regex non ha grandi margini di miglioramento dell'accuratezza:
in sintesi � un approccio rigido, che in caso di inclusione di nuovi pattern testuali 
richiederebbe la continua modifica di una regex in forme sempre pi� complesse, risolvendosi in
problemi di manutenzione, di testing, di leggibilit� e talvolta di sicurezza.
Nel caso del Machine Learning un task di \acrfull{named-entity-recognition} pu� avere risultati migliori delle regex, 
anche se pu� inizialmente soffrire di scarsa numerosit� di entities.
Su questa base si potrebbe proporre l'uso combinato degli output di entrambi i modelli: in particolare,
si potrebbe ricorrere al modello regex per le entit� meno assimilate dal modello \acrfull{machine-learning}
, preferendo invece quest'ultimo per le entit� su cui gli score descrivono prestazioni migliori.

Nell'istogramma in figura \ref{fig:v1-hist} si confrontano le medie di P,R,F1 nel caso della regex e nel caso
dell'NER-ML. Le regex fin qui considerate sono regex non complicate, possono certamente essere migliorate per coprire
una casistica di scritture pi� estesa: prevedendo degli spazi aggiuntivi, prevedendo separatori alternativi e prevedendo 
l'uso combinato di lettere maiuscole e minuscole.
Tuttavia, come sottolineato precedentemente, caratteristica fondamentale dei dati � essere usati nella letteratura 
burocratica: in quest'ambito, colui che scrive il documento formale pu� ad esempio scrivere \say{OG4}, ma anche \say{Opera Generale numero quattro} e ancora \say{Op.Gen.num.4}, per cui risulta ingiustificato investire tempo per ottenere una regex \say{ottimale}.
Un'alternativa potrebbe nascere dalla combinazione dell'uso del NER-regex e del NER-ML.

\includegraphics[clip, width=0.6\textwidth,height=0.6\textheight,keepaspectratio, page=1]{regex-v1-vs-ml.pdf} 
%	\captionof{figure}{confronto tra l'approccio regex(sinistra) e NN(destra)}
\caption{Da sinistra: Valori di P,R,F del NER-regex-v1 e del NER-ML}
\label{fig:v1-hist}


\section{Human In The Loop}
Fin qui il lavoro di \acrshort{societa-soa} extraction ci ha portati a considerare due tecnologie molto diverse per
approccio e manutenibilit�.

In particolare, le \textbf{regex} risultano di risultato immediato poich� in pochi minuti ci consentono di
svolgere un'estrazione di dati SOA con discreti valori di recall,
ma soffrono di \textbf{scarsa leggibilit�}, manutenzione problematica e risultano perci� decisamente \textbf{error-prone} ;
nonostante un incoraggiante risultato immediato, non permettono di migliorare in maniera efficiente l'estrazione di 
entit� per tutta una serie di fattori.
Consideriamo ad esempio il manutentore a cui toccher� estendere la regex in produzione \(reg\_prod\sb{n}\): 
il suo compito sar� tipicamente quello di trasformare la \(reg\sb{prod}\) in una \(reg\_prod\sb{n+1}\)
in maniera da includere i nuovi pattern \( p\sb{m+1} , \ldots , p\sb{m+k} \).
Le domande che ci poniamo sono allora le seguenti:
\begin{itemize}
\item
Quella vecchia regex \(reg\_prod\sb{n}\) sar� stata fino ad allora documentata per tenere traccia dei
pattern precedenti \( p\sb{1}, \ldots ,  p\sb{m} \) ?
\item Oppure spetter� al programmatore ricordare tutti i singoli pattern che venivano catturati con la precedente versione della regex?
\item
E la regex avr� dei dati di \textbf{testing}, per verificare che la nuova versione \(reg\_prod\sb{n+1}\)
non comprometta i pattern fino ad allora accettati?
\end{itemize}
Tutte queste domande si ritrovano nell'indagine statistica di \say{Regexes are hard} \cite{RegexesAreHard}, 
dove emerge che le best practices dell'uso delle regex sono sistematicamente disattese:
le regex vengono scritte perlopi� senza aggiungere documentazione e senza dati di test.
Tutte queste problematiche nel complesso impediscono ad un sistema  basato su regex di essere efficiente nel miglioramento della \acrlong{Precision}.

Al contrario, le reti neurali possono nel tempo \say{imparare meglio} le entit� con cui hanno a che fare ma richiedono un consistente lavoro umano di annotazione e classificazione delle entit� presenti nei dati stessi;
in altre parole, mancano dell'immediatezza delle regex ma rendono fattibile il \textbf{fine-tuning} del sistema classificatore.

%\includegraphics[trim=0 300 0 0, clip, width=\textwidth,height=\textheight,keepaspectratio, page=1]{grafo_regex.pdf}
%\includegraphics[trim=0 300 0 0, clip, width=\textwidth,height=\textheight,keepaspectratio, page=1]{grafo_nn.pdf}
%\includegraphics[trim=0 300 0 0, clip, width=\textwidth,height=\textheight,keepaspectratio, page=1]{grafo_regexnn.pdf} 

I rispettivi punti di forza e di debolezza delle due metodologie risultano complementari, per cui ha senso combinarli in un
unico sistema che sfrutti i vantaggi di entrambe, mitigandone gli svantaggi.

% HIL DIAGRAM
\includegraphics[trim= 10 200 0 100, clip, width=\textwidth,height=\textheight,keepaspectratio, page=1]{flow_chart_hil.pdf} 
\caption{Framework HITL: fase di addestramento} 
\begin{comment}

	\begin{figure}[ht] % placed here or on the top of page
		\begin{tikzpicture}[node distance=3cm]
		%\begin{tikzpicture}%[->,shorten >=1pt,auto,node distance=2.8cm,semithick]
			\node (data) [state] {Data};
			\node (module0) [process, below=1cm of data] {regex\_ner};
			\draw [arrow] (data) -- (module0);
			\node (wl) 		[state, below=1cm of module0] {wl};
			\draw [arrow] (module0) -- (wl);
			\node (humcorr) [process, below=1cm of wl] {human\_correc};
			\draw [arrow] (wl) -- (humcorr);
			\node (l) 		[state, left of=humcorr] {labels};
			\draw [arrow] (humcorr) -- (l);
			\node (train) 	[process, below of=l] {training};
			\draw [arrow] (l) -- (train);
			\node (nn) 		[state, below of=train] {nn};
			\draw [arrow] (train) -- (nn);
			\node (module) [process, right of=nn] {nn\_usage};
			\draw [arrow] (nn) -- (module);
			\node (entities) [state, below of=module] {Entities};
			\node (entitiescorr) [state, above of=module] {Entities};
			\draw [arrow] (module) -- (entities);
			\draw [arrow] (module) -- (entitiescorr);
			\draw [arrow] (entitiescorr) -- (humcorr);
			
\begin{pgfonlayer}{background}
\filldraw [fill=black!30,draw=red] (3,-5) rectangle (-5,-13.5);
%\filldraw [fill=black!30,draw=red] (nn.south -| nn.west) rectangle (humcorr.north -| entitiescorr.east);
\end{pgfonlayer}
		\end{tikzpicture}
		\caption{Human In The Loop framework} % caption
		%\label{fig:diagram}             % for referencing of figure, key select as you wish
	\end{figure}
\end{comment}



Configuriamo l'uso delle componenti regex, \acrshort{machine-learning}, task di creazione regex e task di 
annotazione manuale come un sistema di tipo \acrshort{human-in-the-loop}.
In tale sistema l'essere umano pu� essere utilizzato come programmatore di regex e come annotatore a pi� riprese successive.
Partendo dallo studio \say{How to invest my time} \cite{HowToInvestMyTime}, l'indicazione che traiamo per usare al meglio
il lavoro umano � investire pochi minuti sulla creazione di regex ad alta \acrlong{Recall} per poi effettuare la correzione umana delle annotazioni prodotte dalla regex stessa.
Seguendo la convenzione dello studio citato, chiameremo \say{weak labelling} l'estrazione svolta con la regex, che indicheremo con \(RE\sb{WL}\): le label cos� prodotte risultano \say{weak}, ossia non verificate e passibili d'errore.
%La \textbf{human annotation} prender� in input le \textbf{weak labels} prodotte con la  \(RE\sb{WL}\)
%e produrr� annotazioni che potremo considerare affidabili perch� verificate da un umano;
%tali annotazioni costituiranno il training set della rete neurale.
L'attivit� umana \textbf{\(correct\sb{k}labels\)}, partendo dalle weak labels come input iniziale \(labels\sb{0}\), ad ogni iterazione \textit{i} apporter� \textit{k} correzioni che andranno a formare l'insieme di labels \(labels\sb{i}\). Qualora il ciclo continui, tale insieme i-esimo potr� essere adoperato come input dell'iterazione \textit{i+1}; in caso contrario verr� assunto come training set per istruire la rete neurale.


%\begin{left}
\includegraphics[ clip, width=\textwidth,height=\textheight,keepaspectratio]{PlotFunctionData_30+.pdf} 
\caption{Human In The Loop: F1-score, Precision e Recall.} 
\label{fig:hitl}
%\end{left}

L'uso dello \acrlong{human-in-the-loop} pu� essere vantaggioso se effettuato in maniera tale da:
\begin{itemize}
\item istruire una rete neurale in un tempo minimo di lavoro umano;  
\item permettere il fine-tuning della rete neurale stessa
con l'intervento dell'annotatore umano.
\end{itemize}

Seguendo dunque lo studio gi� citato, l'operatore umano deve destinare pochi minuti alla realizzazione delle  regex \(RE\sb{WL}\). Nella nostra prova utilizziamo le regex \[regex-soa\sb{wl-v1}=\textsc{\char13}O(S|G)(\\d\\d?)\textsc{\char13} \]  \[ regex-class\sb{wl-v1}=\textsc{\char13}(IV|IV|III|II|IX|VIII|VII|VI|X|V|I)\textsc{\char13} \] realizzate in pochi minuti su un tool di regex-editing liberamente disponibile online. Usando tali regex abbiamo quindi prodotto le weak labels che vanno a formare il dataset iniziale \(labels\sb{0}\);
per ogni iterazione successiva \(i>0\) vengono 
 dunque corrette \textit{k} annotazioni weak con \(k=100\). Il grafico \ref{fig:hitl} illustra l'evoluzione del sistema \acrshort{human-in-the-loop} all'aumentare 
delle correzioni umane apportate alle weak labels: viene
rappresentato il guadagno in termini di \acrlong{Precision}, \acrlong{Recall} e \acrlong{F1-score}.
In vari esperimenti analoghi a quello riportato in figura \ref{fig:hitl} si evince come in dieci iterazioni di correzione il %sistema guadagni oltre il 22\% di \acrlong{Precision}, 34\% di \acrlong{Recall} e 32\% di \acrlong{F1-score}.
sistema valutato in termini di  \acrshort{Precision}, \acrshort{Recall}, \acrshort{F1-score} riporti i guadagni minimi:
\[\Delta\sb{i=10}(P,R,F1)=(22\%,34\%, 32\%)\]
� interessante notare come alla settima iterazione correttiva tali guadagni ammontano gi� al
\[\Delta\sb{i=7}(P,R,F1)=(21\%,33\%, 31\%)\]
Portando il discorso ai risultati specifici dell'esperimento al grafico \ref{fig:hitl}, l'iterazione numero dieci porta ai valori \[[P,R,F]\sb{NER-HITL}=[0.8458, 0.7071, 0.7527 ] \] a fronte dei valori 
del sistema NER-ML costruito sulla \textit{ground truth} \[[P,R,F]\sb{NER-ML}= [0.8476, 0.7778, 0.7973] \]. In figura 
\ref{fig:ist} possiamo vedere graficamente il confronto tra \textit{NER-HITL} e \textit{NER-ML}.%\includegraphics[ clip, width=\textwidth,height=\textheight,keepaspectratio]{PlotFunctionData_hitl_vs_ml.pdf} \caption{} \label{fig:hitlvsml}



\center
\includegraphics[]{PlotFunctionData_hitl_vs_ml_notitle.pdf}
\caption{p,r,f1-score di NER-HITL-v1 e di NER-ML}
\label{fig:ist}
\center
% ML 0.847669252797008	0.777828714411617	0.797399765346802
% p 21 r 33 f 31 segna come iterazione 7
% p 25 r 34 f 32 (_v1_29_n)
% p 22 r 38 f 35 (_v1_30_n+)
% p 25 r 36 f 34 (_v1_30_n)
% hitl p 0.845849682049472 r 0.707145200862464 f 0.752779411712778
% gold p 0.847669252797008 r 0.777828714411617 f 0.797399765346802
