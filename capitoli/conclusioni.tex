% !TEX encoding = IsoLatin 

% Affinch� gli accenti vengano accettati, assicurati che la codifica di questo file
% sia ISO 8859-1

% PER OTTENERE IL PDF, digitare da terminale
% ./makepdfplease
% 

\chapter{Conclusioni}
Scopo di questa tesi � stato effettuare un task di Named Entity Recognition 
su documenti provenienti dalla pubblica amministrazione; tale set di documenti,
rumorosi in un quanto spesso ottenuti via OCR a partire da documenti stampati,
risulta spesso descrivere gli stessi dati formali con diciture variabili.
� stata presentata l'implementazione del task di NER con l'uso di regex,
ma � apparso chiaro come questo approccio risulti difficile da migliorare nel tempo;
infatti le regex alla modifica risultano error-prone e di difficile manutenzione e documentazione,
caratteristiche che non aiutano ad esprimere pattern molto variabili.
D'altro canto � stato presentato un NER che adopera tecniche di supervised deep learning
e che meglio si adegua ad estrarre dati variabili, a spese tuttavia di un
ingente tempo umano di annotazione.
� apparsa dunque conveniente la possibilit� di effettuare una \say{weak supervision}
adoperando un dataset prodotto dal task di NER-regex;
tale dataset viene prodotto in pochi minuti, dopodich� � sottoposto a cicliche iterazioni di correzione
da parte di un annotatore umano. Il loop di addestramento cos� configurato � detto Human In The Loop.
Abbiamo notato come l'operatore umano sia significativamente pi� veloce nel compito di 
\textit{correzione-annotazione} rispetto al compito di produzione di annotazioni \textit{from scratch}.
Valutando le perfomance di NER-hitl-i al variare dell'iterazione i-esima, � emerso che:
� possibile ottenere un risultato subottimale con la sola \textit{annotazione-correzione} di 1000 annotazioni,
sacrificando 7\% di recall ma comunque risparmiando pi� del 70\% del tempo d'annotazione.
Infine, portando le iterazioni di NER-hitl-i a 14 � possibile superare le prestazioni di NER-ML,
adoperando comunque il 10\% di annotazioni in meno e risparmiando un tempo di annotazione pari a 58\%.
� possibile immaginare dei futuri miglioramenti per lo schema Human In The Loop individuato;
in particolare si potrebbe fornire all'annotatore una metrica per calcolare quali porzioni di testo
sono pi� rumorose e probabilmente pi� prioritarie rispetto al task di annotazione-correzione .