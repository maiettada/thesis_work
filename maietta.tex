% !TEX encoding = IsoLatin

% La riga soprastante serve per configurare gli editor TeXShop, TeXWorks
% e TeXstudio per gestire questo file con la codifica IsoLatin o Latin 1
% o ISO 8859-1.


% Affinch� gli accenti vengano accettati, assicurati che la codifica di questo file
% sia ISO 8859-1



% per commentare una riga mettere % al suo inizio
% per s-commentare una riga (ossia attivarla) togliere il % al suo inizio
%
\documentclass[% formato PDF/A, obbligatorio per l'archiviazione delle tesi di Polito
,cucitura%lascia margine per la rilegatura
%,twoside% per stampa fronte-retro (fortemente consigliato per tesi voluminose, opzionale per le altre)
%,12pt% font pi� grande (12pt) rispetto a quello normalmente usato (11pt)
]{toptesi}
%
%\usepackage[a-1b]{pdfx}
\usepackage{hyperref}
\hypersetup{%
    pdfpagemode={UseOutlines},
    bookmarksopen,
    pdfstartview={FitH},
    colorlinks,
    linkcolor={blue},
    citecolor={red},
    urlcolor={blue}
  }
% \documentclass[11pt,twoside,oldstyle,autoretitolo,classica,greek]{toptesi}
% \usepackage[or]{teubner}
%%%%%%%%%%%%%%%%%%%%%%%%%%%%%%%%%%%%%%%%%%%%%%%%%%%%
%
% Esempio di composizione di tesi di laurea.
%
% Questo esempio e' stato preparato inizialmente 13-marzo-1989
% e poi e' stato modificato via via che TOPtesi andava
% arricchendosi di altre possibilita'.
%
% Nel seguito laurea "quinquennale" sta anche per "specialistica" o "magistrale"

% Cambiare encoding a piacere; oppure non caricare nessun encoding se si usano
% solo caratteri a 7 bit (ASCII) nei file d'entrata.
%
\usepackage[latin1]{inputenc}% IMPORTANTE! usare codifica ISO-8859-1 per le lettere accentate

% !TEX encoding = IsoLatin

% per inserire uno spazio "fantasma" nella definizione di un'abbreviazione
\usepackage{xspace}

% per inserire un DOI senza problemi coi caratteri "strani" ivi presenti
\usepackage{doi}
\renewcommand{\doitext}{DOI }% originally was "doi:"

% per inserire correttamente le unit� di misura SI (incluse quelle binarie)
\usepackage[binary-units]{siunitx}
% se si desidera usare / invece che la potenza -1 per indicare "al secondo"
\sisetup{per-mode=symbol}

% per inserire codice di programmazione complesso
\usepackage{listings}% per inserire codice di programmazione complesso
\lstset{
basicstyle=\ttfamily,
columns=fullflexible,
xleftmargin=3ex,
breaklines,
breakatwhitespace,
escapechar=`
}

% modify some page parameters
\setlength{\parskip}{\medskipamount}

% riga orizzontale
\newcommand{\HRule}{\rule{\linewidth}{0.2mm}}
% esempio di creazione di semplici abbreviazioni
\newcommand{\ltx}{\LaTeX\xspace}
\newcommand{\txw}{TeXworks\xspace}
\newcommand{\mik}{MikTex\xspace}
\newcommand{\html}{HTML\xspace}
\newcommand{\xhtml}{XHTML\xspace}

% esempio di creazione di un'abbreviazione con un parametro (il cui uso � indicato da #1)
\newcommand{\cmd}[1]{\texttt{#1}\xspace}
% per citare un RFC, es. \rfc{822}
\newcommand{\rfc}[1]{RFC-#1\xspace}
% per citare un file (es. \file{autoexec.bat}) o una URI fittizia (es. \file{http://www.lioy.it/})
% per le URI vere usare \url o \href
\newcommand{\file}[1]{\texttt{#1}\xspace}
% per inserire codice di esempio in-line
\newcommand{\code}[1]{\lstinline|#1|}
% importante per i pathname Windows perch� non si pu� usare \ essendo un carattere riservato di Latex
\newcommand{\bs}{\textbackslash}
% definizione di un termine: formattazione ed inserimento nell'indice
\newcommand{\tdef}[1]{\textit{#1}\index{#1}}
% meta-termine, usato tipicamente nelle definizioni dei tag
\newcommand{\meta}[1]{\textit{#1}}
% abbreviazioni in inglese
\newcommand{\ie}{i.e.\xspace}
\newcommand{\eg}{e.g.\xspace}
\usepackage{dirtytalk}

\begin{document}

\ateneo{Politecnico di Torino}

%%% scegliere la propria facolt� (solo PRIMA dell'AA 2012-2013)
%
%\facolta[III]{Ingegneria dell'Informazione}
%\facolta[IV]{Organizzazione d'Impresa\\e Ingegneria Gestionale}
%\Materia{Remote sensing}% uso sconsigliato

%\monografia{Gestione informatizzata di un magazzino ricambi}% per la laurea triennale
\titolo{Estrazione di dati della P.A. (v. 1.02)}% per la laurea quinquennale e il dottorato
%\sottotitolo{Metodo dei satelliti medicei}% NON obbligatorio, per la laurea quinquennale e il dottorato

%%% scegliere il proprio corso
%
%\corsodilaurea{Ingegneria dell'Organizzazione d'Impresa}% per la laurea di primo e secondo livello
%\corsodilaurea{Ingegneria Logistica e della Produzione}% per la laurea di primo e secondo livello
%\corsodilaurea{Ingegneria Gestionale}% per la laurea di primo e secondo livello
\corsodilaurea{Ingegneria Informatica}% per la laurea di primo e secondo livello
%\corsodidottorato{Meccanica}% per il dottorato

\candidato{Davide \textsc{Maietta}}% per tutti i percorsi
%\secondocandidato{Evangelista \textsc{Torricelli}}% per la laurea magistrale solamente
%\direttore{prof. Albert Einstein}% per il dottorato
%\coordinatore{prof. Albert Einstein}% per il dottorato
\relatore{prof.\ Maurizio Morisio}% per la laurea e il dottorato
%\secondorelatore{dipl.~ing.~Werner von Braun}% per la laurea magistrale
%\terzorelatore{{\tabular{@{}l}dott.\ Neil Armstrong\\prof. Maria Rossi\endtabular}}% per la laurea magistrale
%\tutore{ing.~Karl Von Braun}% per il dottorato
%\tutoreaziendale{dott.\ ing.\ Giovanni Giacosa} % solo per la laurea di secondo livello con tesi svolta in azienda
%\NomeTutoreAziendale{Supervisore aziendale\\Centro Ricerche FIAT}
%\sedutadilaurea{Agosto 1615}% per la laurea quinquennale
%\esamedidottorato{Novembre 1610}% per il dottorato
\sedutadilaurea{\textsc{Novembre} 2017}% per la laurea triennale
%\sedutadilaurea{\textsc{Anno~accademico} 1615-1616}% per la laurea magistrale
%\annoaccademico{1615-1616}% solo con l'opzione classica
%\annoaccademico{2006-2007}% idem
%\ciclodidottorato{XV}% per il dottorato
\logosede{logopolito}
%
%\chapterbib %solo per vedere che cosa succede; e' preferibile comporre una sola bibliografia
%\AdvisorName{Supervisors}
%\newtheorem{osservazione}{Osservazione}% Standard LaTeX


%\usepackage[a-1b]{pdfx}
%\hypersetup{%
%    pdfpagemode={UseOutlines},
%    bookmarksopen,
%    pdfstartview={FitH},
%    colorlinks,
%    linkcolor={blue},
%    citecolor={green},
%    urlcolor={blue}
%  }
%
% per numerare e far comparire nell'indice anche le sezioni di quarto livello
% SCONSIGLIATO! da usarsi solo in caso di estrema necessit�
%\setcounter{secnumdepth}{4}% section-numbering-depth
%\setcounter{tocdepth}{4}% TOC-numbering-depth (TOC=Table-Of-Content)

%\setbindingcorrection{3mm}

\errorcontextlines=9

\frontespizio
\paginavuota
\newpage
%per sfruttare meglio lo spazio nella pagina
\advance\voffset -5mm
\advance\textheight 30mm

% opzionale, solo se si vuole dedicare la tesi a delle persone care
\begin{dedica}
A mio padre

\textdagger\ A mio nonno Pino
\end{dedica}

\sommario

Inserire qui un breve sommario della tesi.

\ringraziamenti

Opzionali, solo nel caso si sia ricevuto un aiuto speciale e particolarmente rilevante.

%% inserire sempre nella tesi per la laurea di I livello, perch� il nome dei tutori non � indicato sul frontespizio.
%Il lavoro descritto in questa monografia � stato svolto sotto la supervisione
%del Prof. Antonio Lioy (tutore accademico)% inserire sempre il nome del tutore accademico
% e dell'Ing. Mario Rossi (tutore aziendale)% inserire solo se la monografia � relativa ad un tirocinio.
%.

%\tablespagetrue % normalmente questa riga non serve ed e' commentata
%\figurespagetrue % normalmente questa riga non serve ed e' commentata

\indici

\mainmatter

%\chapter{1}
%\input{capitoli/1-introduzione.tex}


\chapter{Estrazione di dati da testi della pubblica amministrazione}
% !TEX encoding = IsoLatin 

% Affinch� gli accenti vengano accettati, assicurati che la codifica di questo file
% sia ISO 8859-1

% PER OTTENERE IL PDF, digitare da terminale
% ./makepdfplease
% 

% !TEX encoding = IsoLatin 

% Affinch� gli accenti vengano accettati, assicurati che la codifica di questo file
% sia ISO 8859-1

% PER OTTENERE IL PDF, digitare da terminale
% ./makepdfplease
% 
% !TEX encoding = IsoLatin 

% Affinch� gli accenti vengano accettati, assicurati che la codifica di questo file
% sia ISO 8859-1

% PER OTTENERE IL PDF, digitare da terminale
% ./makepdfplease
% 



\chapter{Estrazione di dati da testi della pubblica amministrazione}
%\chapter[Estrazione di dati da testi della pubblica amministrazione][]{1. Estrazione di dati da testi della pubblica amministrazione}
%IoT \cite{friis} basically refers to ....\\
%\ac{IoT} basically refers to ....\\

%\section{Introduzione}
L'avanzamento tecnologico delle telecomunicazioni negli ultimi decenni ha permesso alle persone di recuperare dati e informazioni provenienti dai luoghi pi� disparati e ad una velocit� di approvvigionamento praticamente nulla: se prima di Internet l'unico modo di ricevere un documento ufficiale era farne richiesta all'ufficio competente, ora le documentazioni possono essere normalmente a disposizione in rete. Gli utenti della rete hanno a disposizione una mole cos� grande di documenti e informazioni che gli � praticamente impossibile visionare la totalit� di questi dati.
In questo contesto l'esigenza di estrarre dati da documenti � sempre pi� sentita, soprattutto quando si voglia fare una cernita dei documenti contenenti specifiche informazioni di proprio interesse.
Questa tesi ha come scopo effettuare un'estrazione di dati da un insieme di documenti della \acrfull{Pubblica Amministrazione}; pi� precisamente i documenti appartengono al contesto dei bandi pubblici e i dati da cercare sono tutte quelle informazioni che impongono dei requisiti ai possibili partecipanti di gara. 
In questo contesto si provano approcci tecnologici diversi, nella fattispecie l'uso delle regular expressions e delle reti neurali; di entrambe le tecnologie si delineano punti di forza e svantaggi, per poi investigare se sia preferibile l'una delle due o se invece sia possibile un compromesso che valorizzi il meglio di entrambe. 
Nella sezione che segue delineeremo in dettaglio 

%\section{Contesto}

\section{Gare d'appalto}
Tra le funzioni di uno Stato vi sono la costruzione, la manutenzione e la messa in sicurezza delle infrastrutture di pubblica utilit�, quali ad esempio gli edifici pubblici, le reti di telecomunicazione, le strade e gli ospedali di cui beneficiano i cittadini.
La \acrlong{Pubblica Amministrazione}, dovendo garantire una quantit� cos� grande di infrastrutture, non esegue i lavori direttamente, ma ne delega l'attuazione pratica a delle imprese private con un appalto pubblico.
L'appalto � un contratto con il quale una parte, detta parte appaltatrice, assume, \say{con organizzazione dei mezzi necessari e con gestione a proprio rischio, l'obbligazione di compiere in favore di un'altra (committente o appaltante) un'opera o un servizio}\cite{basacchi2013codice}.
Gli appalti pubblici sono dunque un modo per eseguire delle opere pubbliche pagando delle imprese private per la realizzazione vera e propria; la \acrshort{Pubblica Amministrazione} deve per� assicurarsi che:
l'impresa abbia le conoscenze per portare a termine l'opera rispettando degli standard tecnici, ossia seguendo la \textbf{regola d'arte};
l'impresa gestisca i lavori in maniera competitiva, senza comportare sprechi ingiustificati di soldi pubblici;
l'impresa riesca a reggere lo sforzo finanziario che l'opera comporta, sia per quanto riguarda l'approvvigionamento dei materiali, sia per quanto riguarda il costo del lavoro.
L'assegnazione di un progetto si avvale di apposita gara d'appalto, che ha come obbiettivo la selezione dell'impresa che meglio possa soddisfare questi requisiti di competenze tecniche, finanziarie e di competitivit� sul mercato.

\section{Identificativi di progetto e di gara}
Nell'ambito di una gara d'appalto, il progetto pu� essere composto da pi� parti singole chiamate lotti; l'assegnazione di un lotto � indipendente dagli altri, per cui diversi lotti di un progetto possono essere assegnati ad aziende differenti nell'ambito della stessa gara d'appalto.
I documenti che andiamo ad analizzare riportano una serie di informazioni di nostro interesse, quali:
\begin{itemize}
\item il \textbf{\acrfull{Codice Unico di Progetto}}, che descrive univocamente il progetto d'investimento pubblico;  � un identificativo alfanumerico di quindici caratteri;
\item il \textbf{\acrfull{Codice Identificativo di Gara}}, che indica in maniera univoca il lotto;
 � un identificativo alfanumerico di dieci caratteri;
\end{itemize}

\section{Lavori e importi}
La gara d'appalto deve indicare in maniera evidente e inoppugnabile quali competenze tecniche e quali capacit� finanziarie sono ritenute requisiti fondamentali per la partecipazione; qualsiasi ambiguit� nei requisiti di una gara potrebbe dare adito a ricorsi e quindi a rallentare la gara stessa. 
Il bisogno di requisiti oggettivi ha motivato l'introduzione di apposite certificazioni fornite dalle 
\acrfull{societa-soa}, che prendono il nome di \say{certificati SOA} o \say{attestazioni SOA}.
Tali attestazioni si dividono in due macrogruppi:
\begin{itemize}
\item \textbf{Categorie di lavori:} individuano le categorie di opere dal punto di vista tecnico;
\item \textbf{Classifiche di importi:} individuano i livelli di capacit� finanziaria.
\end{itemize}
Un bando di gara esprimer� i requisiti di progetto sotto forma di categorie e classifiche SOA, per cui un'azienda che voglia  essere ammessa alla gara dovr� possedere le attestazioni SOA richieste.









\section{Tassonomia SOA}
Le Categorie sono suddivise in due macro-categorie: opere generali e opere specializzate,  rispettivamente identificate dagli acronimi \say{OG} e \say{OS}; sono individuate 13 categorie di Opere Generali e 39 categorie di Opere Specializzate.
Le Classifiche d'importo, in numero di dieci, sono rese come numeri ordinali romani. 


\subsection{Categorie di opere generali}\label{CATEGORIE}
\begin{itemize}
  \item \acrshort{OG-1}, \acrlong{OG-1}
  \item \acrshort{OG-2}, \acrlong{OG-2}
\item \acrshort{OG-3}, \acrlong{OG-3}
\item \acrshort{OG-4}, \acrlong{OG-4}
\item \acrshort{OG-5}, \acrlong{OG-5}
\item \acrshort{OG-6}, \acrlong{OG-6}
\item \acrshort{OG-7}, \acrlong{OG-7}
\item \acrshort{OG-8}, \acrlong{OG-8}
\item \acrshort{OG-9}, \acrlong{OG-9}
\item \acrshort{OG-10}, \acrlong{OG-10}
  \item \acrshort{OG-11}, \acrlong{OG-11}
  \item \acrshort{OG-12}, \acrlong{OG-12}
  \item \acrshort{OG-13}, \acrlong{OG-13}
\end{itemize}

\subsection{Categorie di opere specializzate}
\begin{itemize}
  \item \acrshort{OS-1}, \acrlong{OS-1}
  \item \acrshort{OS-2-A}, \acrlong{OS-2-A}
  \item \acrshort{OS-2-B}, \acrlong{OS-2-B}
  \item \acrshort{OS-3}, \acrlong{OS-3}
  \item \acrshort{OS-4}, \acrlong{OS-4}
  \item \acrshort{OS-5}, \acrlong{OS-5}
  \item \acrshort{OS-6}, \acrlong{OS-6}
  \item \acrshort{OS-7}, \acrlong{OS-7}
  \item \acrshort{OS-8}, \acrlong{OS-8}
  \item \acrshort{OS-9}, \acrlong{OS-9}
  \item \acrshort{OS-10}, \acrlong{OS-10}
  \item \acrshort{OS-11}, \acrlong{OS-11}
  \item \acrshort{OS-12-A}, \acrlong{OS-12-A}
  \item \acrshort{OS-12-B}, \acrlong{OS-12-B}
  \item \acrshort{OS-13}, \acrlong{OS-13}
  \item \acrshort{OS-14}, \acrlong{OS-14}
  \item \acrshort{OS-15}, \acrlong{OS-15}
  \item \acrshort{OS-16}, \acrlong{OS-16}
  \item \acrshort{OS-17}, \acrlong{OS-17}
  \item \acrshort{OS-18-A}, \acrlong{OS-18-A}
  \item \acrshort{OS-18-B}, \acrlong{OS-18-B}
  \item \acrshort{OS-19}, \acrlong{OS-19}
  \item \acrshort{OS-20-A}, \acrlong{OS-20-A}
  \item \acrshort{OS-20-B}, \acrlong{OS-20-B}
  \item \acrshort{OS-21}, \acrlong{OS-21}
  \item \acrshort{OS-22}, \acrlong{OS-22}
  \item \acrshort{OS-23}, \acrlong{OS-23}
  \item \acrshort{OS-24}, \acrlong{OS-24}
  \item \acrshort{OS-25}, \acrlong{OS-25}
  \item \acrshort{OS-26}, \acrlong{OS-26}
  \item \acrshort{OS-27}, \acrlong{OS-27}
  \item \acrshort{OS-28}, \acrlong{OS-28}
  \item \acrshort{OS-29}, \acrlong{OS-29}
  \item \acrshort{OS-30}, \acrlong{OS-30}
  \item \acrshort{OS-31}, \acrlong{OS-31}
  \item \acrshort{OS-32}, \acrlong{OS-32}
  \item \acrshort{OS-33}, \acrlong{OS-33}
  \item \acrshort{OS-34}, \acrlong{OS-34}
  \item \acrshort{OS-35}, \acrlong{OS-35}
\end{itemize}

\subsection{Classifiche importo }\label{CLASSIFICHE}
\begin{itemize}
  \item \acrshort{I}, \acrlong{I}
  \item \acrshort{II}, \acrlong{II}
  \item \acrshort{III}, \acrlong{III}
  \item \acrshort{III-bis}, \acrlong{III-bis}
  \item \acrshort{IV}, \acrlong{IV}
  \item \acrshort{IV-bis}, \acrlong{IV-bis}
  \item \acrshort{V}, \acrlong{V}
  \item \acrshort{VI}, \acrlong{VI}
  \item \acrshort{VII}, \acrlong{VII}
  \item \acrshort{VIII}, \acrlong{VIII}
\end{itemize}  

\label{chap:1}


\chapter{Individuazione delle attestazioni SOA}
% !TEX encoding = IsoLatin 

% Affinch� gli accenti vengano accettati, assicurati che la codifica di questo file
% sia ISO 8859-1

% PER OTTENERE IL PDF, digitare da terminale
% ./makepdfplease
% 





\section{Casi d'uso reali}
\subsection{}
Finora sono state descritte le attestazioni SOA ed � stato chiarito il ruolo fondamentale che queste certificazioni ricoprono negli appalti.
D'ora in avanti si intende descrivere l'uso di queste attestazioni da parte dei vari attori del mondo degli appalti:
\begin{enumerate}
  \item la \textbf{Pubblica Amministrazione}: produce il bando di gara, relativo ad un progetto identificato univocamente dal codice CUP ed eventualmente diviso in lotti, ognuno identificato da un codice CIG; nel bando di gara aggiunge i requisiti di Categorie SOA e le Classifiche economiche SOA, imponendo eventualmente vincoli temporali su tali certificati (?L'azienda deve risultare in possesso di tale certificato dal 2020?); il bando viene scritto in formato pdf e pubblicato sul sito dell'ente pubblico ( Comune, Provincia, Regione, Ministero, et cetera), che risulta in questo contesto essere l'appaltante;
\item l'\textbf{imprenditore}, o qualsivoglia candidato appaltatore: naviga i siti web della Pubblica Amministrazione alla ricerca di bandi di gara; per ogni bando di gara trovato, deve comprendere i requisiti SOA e accertarsi di poterli soddisfare; solo se dotato di idonea attestazione, potr� candidarsi ad essere appaltatore prendendo parte alla gara d'appalto.
\end{enumerate}
Sia chiaro che la P.A. pubblica anche altre tipologie di documenti, per cui � utile precisare che ci riferiamo a quell'insieme di documenti che descrivono gare e verbali d'appalto:  possiamo riferirci a questo insieme chiamandolo dominio d'appalto.
In questi due casi d'uso possiamo evidenziare una prima problematica: ogni ente pubblico ha un proprio sito web e la raccolta di documenti di appalti pu� essere un'attivit� lunga e dispersiva. Questo problema potrebbe essere risolto da un software di tipo web-crawler che navighi i siti web della pubblica amministrazione e raccolga tutti i documenti del dominio d'appalto in un dataset.
Esiste una seconda problematica da porre: ammesso che abbia a disposizione tutto il dataset del dominio degli appalti, l'imprenditore che � alla ricerca di un bando adatto alla sua specifica certificazione si trover� costretto a verificare per ogni documento se tale certificazione sia sufficiente per essere ammesso al bando; detto in altre parole, dovr� leggere e annotare manualmente tutti i documenti in base alle attestazioni SOA per poi scegliere il bando con l'annotazione SOA desiderata.
Questa problematica � decisamente noiosa, sia perch� la lettura toglie alla persona un ingente tempo di lavoro, sia perch� un compito del genere � perfettamente automatizzabile; d'ora in poi ci focalizzeremo su questa attivit� specifica.
L'idea di base � permettere alle imprese di migliorare la ricerca di possibili appalti rendendo questi documenti digitalmente navigabili in base alle attestazioni SOA: 
per l'imprenditore il caso d'uso 2 si ridurrebbe a cercare l'attestazione SOA desiderata per avere tutti e soli i bandi che la contemplino tra i requisiti.
La ricerca di attestazioni SOA in un testo � un problema di \textbf{Entity Extraction} (\textbf{EE}).
\paragraph{Formulazione del problema:}
\textit{
Dato un insieme di documenti in formato testuale appartenenti al dominio delle gare d'appalto, estrarre le attestazioni SOA ivi menzionate.
}

\section{Impostazione del problema}
\subsection{Insiemi di valori}
L'impostazione del problema inizia con la definizione degli insiemi di valori possibili;
nel caso delle attestazioni SOA, seguendo la tassonomia esposta definiamo l'insieme delle Categorie e l'insieme delle Classifiche:
\begin{center}
Categorie = \{\say{OG-1}, \say{OG-2}, \say{OG-3}, \say{OG-4},
 \say{OG-5},\say{OG-6},\say{OG-7},\say{OG-8},
 \say{OG-9},\say{OG-10},\say{OG-11},\say{OG-12},
 \say{OG-13}, \say{OS-1},\say{OS-2A},\say{OS-2B},
 \say{OS-3},
 
 \say{OS-4},\say{OS-5},\say{OS-6},
 \say{OS-7},\say{OS-8},\say{OS-9},\say{OS-10},
 \say{OS-11},\say{OS-12A},
 
 \say{OS-12B}, \say{OS-13},
 \say{OS-14},\say{OS-15},\say{OS-16},\say{OS-17},
 \say{OS-18A},\say{OS-18B},
 
 \say{OS-19},
 \say{OS-20A},
 \say{OS-20B},\say{OS-21},\say{OS-22},\say{OS-24},
 \say{OS-25}, \say{OS-26}, 
 
 \say{OS-27}, \say{OS-28},
 \say{OS-29}, \say{OS-30}, \say{OS-31}, \say{OS-32},
 \say{OS-33}, \say{OS-34}, 
 \say{OS-35} \}
\end{center}

\begin{center}
Classifiche = \{\say{I}, \say{II}, \say{III}, \say{IV}, \say{IV-bis}, \say{V}, \say{VI}, \say{VII}, \say{VIII} \};
\end{center}

In definitiva, il sistema indicher� le informazioni estratte in base all'insieme Categorie e Classifiche cos� definiti. 
\subsection{Formato dei dati estratti}
Pi� in generale, per ogni attestazione SOA che il sistema individuer� in un documento, dovr� indicare varie informazioni:
\begin{itemize}
    \item la porzione di testo in cui l'attestazione SOA � individuata;
    \item la categoria riconosciuta;
    \item la classifica riconosciuta;
    \item lo score, ossia il grado di accuratezza dell'output.
\end{itemize}

\paragraph{soa\_extracted\_tuple} = [  (start\_offset\_cat, end\_offset\_cat), (categoria, classifica), score)
            ]

\section{Valutazione dei dati estratti}
Quando il sistema collega la stringa "o.g. 1" all'elemento OG-1, sta effettuando una classificazione: sta classificando un dato come appartenente alla classe delle istanze di OG-1.
Il sistema di estrazione dati effettua una classificazione su ogni porzione di testo individuata; ogni porzione di testo pu� essere collegata alla classe di uno dei valori SOA.
Tipicamente, in un problema di classificazione l'output pu� essere valutato come segue:
\begin{itemize}
    \item True Positive(TP), se l'elemento � stato assegnato correttamente alla classe;
    \item False Positive(FP), se l'elemento � stato assegnato erroneamente alla classe;
    \item True Negative(TN), se l'elemento non � stato assegnato alla classe perch� non vi andava assegnato;
    \item False Negative(FN), se l'elemento non � stato assegnato alla classe ma vi andava assegnato.
\end{itemize}

Ogni output del sistema verr� valutato in termini di TP-FP-TN-FN; questo permetter� di calcolare la bont� del sistema in termini di:

\paragraph{Precision} = (TP)/(TP + FP )
\paragraph{Recall} = (TP)/(TP + FN )
\paragraph{Accuracy} = (TP + TN)/(TP + TN + FP + FN)

\section{Ground truth}
Per ogni istanza di testo assegnato ad una classe, bisogna affermare se la classificazione data costituisce true positive, false positive, true negative o false negative; questo vuol dire che, per valutare la bont� di una classificazione, abbiamo bisogno di stabilire quale sia la classificazione giusta.
Il sistema che stiamo considerando effettua un'estrazione di dati, classificandoli come elementi delle attestazioni SOA; per valutare la bont� del sistema abbiamo dunque bisogno di stabilire per ogni porzione di testo l'output giusto, ovvero di definire la \textbf{verit�} in base alla quale valutare il sistema.
La \textbf{Ground Truth} � dunque definita come l'output corretto che ci si aspetterebbe dal sistema; e come tale indicher�, per ogni porzione di documento, l'attestazione "giusta" che il sistema avrebbe dovuto estrarre.
La ground truth verr� poi confrontata con l'output del sistema e permetter� di classificare ogni tupla dell'output come true positive, false positive, true negative, false negative.
Una volta contrassegnata ogni tupla dell'output come tp/fp/tn/fn, sar� possibile calcolare Precision, Recall e Accuracy.

\section{Tecniche di estrazione}
Di seguito si espongono due approcci principali per l'estrazione di dati:
\begin{itemize}
    \item l'uso di regular expression;
    \item l'uso di tecniche di machine learning.
\end{itemize}
Mostreremo che hanno differenti punti di forza, differenti punti di debolezza e confronteremo i risultati di entrambe le tecnologie.
\label{chap:2}

\chapter{Lavori correlati}
% !TEX encoding = IsoLatin 

% Affinch� gli accenti vengano accettati, assicurati che la codifica di questo file
% sia ISO 8859-1

% PER OTTENERE IL PDF, digitare da terminale
% ./makepdfplease
% 


\chapter{Estrazione di entit� da testi}
%\chapter[Impiego delle regular expression][]{3. Impiego delle regular expression} %evita la dicitura "capitolo"
In questo capitolo vogliamo focalizzarci sulla \acrfull{named-entity-recognition} e sulle varie possibilit� di estrarre 
e classificare informazioni contenute in documenti di testo. L'estrazione di pattern testuali � una pratica consolidata 
nella programmazione tradizionale che fa abbondante uso delle \textbf{regular expression}; tuttavia, le regex soffrono di una
serie di problematiche che le rendono uno strumento poco affidabile se usato da solo. Tali problematiche possono essere arginate e 
risolte se alle regex si affiancano altre tecnologie.
Un altro strumento usato in NER sono le reti neurali o \acrfull{neural-network}, che a differenza delle regex non \say{apprendono}
un pattern testuale con una grammatica regolare, ma con l'uso di un quantitativo massiccio di esempi di dati \textbf{di training} opportunamente classificati, con un approccio che in \acrlong{machine-learning} viene definito \textbf{supervisionato}.
Vedremo che la \acrlong{named-entity-recognition} pu� avvalersi anche di una tecnologia ibrida che sfrutti in contemporanea strumenti
diversi, tra cui regular expressions, ontologie e reti neurali.
Infine un pattern degno di nota � lo \acrfull{human-in-the-loop}, che permette l'intervento di un revisore umano per apportare correzioni
alle entit� estratte dal sistema NER; tale pattern si rivela particolarmente vantaggioso perch� abilita il sistema NER ad \textbf{apprendere} le correzioni apportate per migliorare le prestazioni successive.



\section{NER con Regex}
Le regex sono notoriamente un potente strumento software, ma risultano difficilmente leggibili, i loro usi pratici sono documentati poco  e si rivelano poco manutenibili; la comunit� dei programmatori ha persino coniato il detto \textit{"Now you have two problems"}\cite{nowTwoProblems}, che esprime come le soluzioni software basate su regex, lungi dall'essere considerate affidabili, creino ulteriori problemi 
a causa della gestione delle regex stesse.
Nello studio \textit{"How to invest my time"} \cite{HowToInvestMyTime} gli autori, ben consapevoli di quanto le regex siano complesse 
ed error-prone, si domandano \textit{fino a che punto} possano essere usate vantaggiosamente;
pi� in particolare il loro studio si cala nel contesto della Entity Extraction e si pone l'obiettivo di usare al 
meglio le risorse umane, studiando due attivit� diverse e complementari:
\begin{enumerate}
\item \label{here:att1}lo sviluppo di regex per produrre automaticamente annotazioni dati testuali;
\item \label{here:att2}l'annotazione manuale delle entit� contenute nei dati stessi.
\end{enumerate}
Le due attivit� sono compiute da operatori \textbf{umani}, per cui gli autori con questo studio hanno voluto indagare come
utilizzare al meglio il tempo di un dipendente annotatore e programmatore, provando varie combinazioni delle due attivit� menzionate.
La regex prodotta al punto ~\ref{here:att1} genera annotazioni dette \textbf{weak labels}, che vanno a costituire il \textbf{training set} della
\acrlong{neural-network}. Al punto  ~\ref{here:att2} l'annotatore pu� creare annotazioni \textit{ex novo} per ampliare il dataset di training, ma pu�
anche effettuare attivit� di \textbf{fine tuning}, correggendo le weak labels che la regex ha generato.
Le due azioni concorrono a formare, addestrare e infine perfezionare una \acrlong{neural-network} e sono state sperimentate con differenti modalit� temporali,
creando scenari in cui il tempo � stato speso in diverse proporzioni sulla prima o sulla seconda attivit�.
I risultati sperimentali di questo studio mostrano che:
\begin{itemize}
\item se il tempo da investire nella EE � poco (inferiore ai 40 minuti), conviene che l'operatore si limiti a produrre una regex;
\item se il tempo � molto (superiore ai 40 minuti), l'operatore potrebbe spendere tutto il tempo a sua disposizione per creare annotazioni con cui istruire la rete neurale;
\item tra i due casi estremi, pu� convenire che l'operatore umano spenda pochi minuti per creare una regex per un primo setup di rete neurale, per poi aggiungervi annotazioni manuali per farne fine-tuning.
\end{itemize}
Questo approccio che contempla le azioni umane in un sistema automatico da addestrare e perfezionare � detto \textbf{Human In The Loop}.  

\section{Deep Learning con Human-In-The-Loop}
Nel campo della Named Entity Recognition(NER) i metodi di Deep Learning hanno un discreto successo,
perch� richiedono un'ingegnerizzazione limitata \cite{ImprovingNE}; allo stesso tempo 
per� hanno bisogno di grandi quantit� di dati per effettuare il training dei modelli.
Lo studio Improving Named Entity Recognition propone quindi un uso iterativo degli annotatori umani all'interno del sistema Human NERD (dove NERD sta per Named Entity Recognition with Deep Learning).
Questo sistema di \textbf{Human In The Loop} si articola cos�:
\begin{enumerate}
\item Viene raccolto un dataset T di documenti non annotati;
\item Al sistema Human NERD vengono forniti modelli NER esterni da acquisire; importando tali modelli, Human NERD effettua una prima annotazione dei documenti del dataset T; 
\item Human NERD invia ciascun documento preannotato ad un annotatore umano; poich� si prevede che gli annotatori possano essere in numero maggiore di uno, ogni documento apparterr� esclusivamente ad un annotatore;
l'annotatore prescelto effettua la review delle annotazioni, aggiungendo, rimuovendo o correggendo etichette.
Il risultato della revisione viene inviato al framework;
\item Basandosi sulle correzioni ricevute, Human NERD pu� aggiornare il modello in maniera incrementale; in alternativa pu� fare training da zero, istruendo cos� un modello nuovo.
\item Basandosi sui cambiamenti effettuati, Human NERD calcola il nuovo livello di Accuracy, computa il numero di occorrenze per classe d'entit�, calcola il loss basato sulle attivit� di training e labelling; infine calcola una stima del gain dovuta al miglioramento dell'accuracy.
\end{enumerate}
In definitiva l'approccio \textbf{\acrlong{human-in-the-loop}} � senz'altro promettente per la costruzione di dataset e per il training di modelli \acrshort{named-entity-recognition} sempre pi� accurati.

\section{Liste di dati da documenti sottoposti a OCR}
Altro interessante contributo al Natural Language Processing � lo studio di Packer et al.\cite{costeffective},
in cui si propone il funzionamento del software ListReader per l'acquisizione di dati da documenti sottoposti ad OCR.
Pi� nello specifico ListReader acquisisce \textbf{liste} di dati; spetta ad un utente utilizzare l'interfaccia grafica 
di ListReader per compilare un generico form che descriva la struttura dei dati: ad esempio, si pensi ad una lista di persone; ogni elemento della lista dovr� fornire il Nome e il Cognome della persona ed eventualmente altre informazioni anagrafiche; l'utente che osserver� tali dati li user� per creare un form contenente i textbox "Nome", "Cognome", "Data di nascita", "Data di Decesso", etc.
Da questa parziale annotazione ListReader potr�:
\begin{enumerate}
\item popolare un'ontologia con entit� e attributi derivanti dal form;
\item indurre una regex per ogni dato identificato, costruendo un wrapper di regex iniziali;
\item generalizzare le regex iniziali con un algoritmo A*, per renderle adeguate a catturare anche dati pi� complessi;
\item fare \textbf{active learning}: consentire all'utente di modificare il form iniziale quando si presentino dati non aderenti alla struttura descritta dal form.
\end{enumerate}
Infine ListReader risulta pi� performante degli algoritmi CRF per l'acquisizione di elementi da liste.




\label{chap:3}

%\chapter{cap4}
%\input{capitoli/Chapter2.tex}

%\chapter{5}
%\input{capitoli/5-design.tex}

%\chapter{6}
%\input{capitoli/6-implementazione.tex}

%\chapter{7}
%\input{capitoli/7-testing.tex}

%\chapter{8}
%\input{capitoli/8-conclusioni.tex}



% bibliografia scritta "a mano"
%% !TEX encoding = IsoLatin

% La bibliografia, da inserirsi solo se ci sono state citazioni.
% In questo caso ricordarsi che bisogna sempre elaborare due volte il file .TEX
% perch� la prima volta viene generata la bibliografia mentre la seconda volta viene inclusa

% NOTA: citare il DOI non � obbligatorio ma MOLTO desiderabile

\begin{thebibliography}{9} % se ci sono meno di 10 citazioni
%\begin{thebibliography}{99} % se ci sono da 10 a 99 citazioni
%\begin{thebibliography}{999} % se ci sono da 100 a 999 citazioni

% esempio citazione articolo a congresso
\bibitem{psisec}
% nomi autori
I.Enrici, M.Ancilli, A.Lioy,
% titolo articolo
``A psychological approach to information technology security'',
% nome del congresso
HSI-2010: 3rd Int. Conf. on Human System Interactions,
% luogo (stato) e data del congresso
Rzesz�w (Poland), May 13-15, 2010,
% pagine dell'articolo
pp.\ 459-466,
% DOI
\doi{10.1109/HSI.2010.5514528}

% esempio citazione articolo su rivista
\bibitem{tpa}
% autori dell'articolo
G.Cabiddu, E.Cesena, R.Sassu, D.Vernizzi, G.Ramunno, A.Lioy,
% titolo dell'articolo
``Trusted Platform Agent'',
% nome della rivista
IEEE Software,
% volume e numero della rivista (alcune riviste non ce l'hanno)
Vol.\ 28, No.\ 2,
% mese e anno di pubblicazione della rivista
March-April 2011,
% pagine dell'articolo
pp.\ 35-41,
% DOI
\doi{10.1109/MS.2010.160}


% esempio citazione capitolo di un libro fatto come collezione di contributi da autori diversi
\bibitem{tc}
A.Lioy, G.Ramunno, % autori del capitolo
``Trusted Computing'' % titolo del capitolo
nel libro % in the book
``Handbook of Information and Communication Security'' % titolo del libro
a cura di % edited by
P.Stavroulakis, M.Stamp, % nomi dei curatori
Springer, % nome editore
2010, % anno di pubblicazione
pp.\ 697-717, % pagine del capitolo
\doi{10.1007/978-3-642-04117-4_32}

 % esempio citazione pagina web di un progetto
\bibitem{openssl}
% nome del progetto
The OpenSSL project,
 % URI della pagina web
\url{http://www.openssl.org/}

% esempio citazione RFC
\bibitem{tls12}
T.Dierks, E.Rescorla,
``The Transport Layer Security (TLS) Protocol Version 1.2'',
\rfc{5246}, August 2008,
\doi{10.17487/RFC5246}

 % esempio citazione libro
\bibitem{seceng}
Ross J. Anderson,
``Security engineering'',
Wiley, 2008,
ISBN: 978-0-470-06852-6

\end{thebibliography}


% se la bibliografia � stata scritta (usando Bibtex) nel file biblio.bib allora commentare la riga precedente e scommentare le due righe seguenti
\bibliographystyle{torsec}
\bibliography{biblio.bib}



\end{document}
